\documentclass[a4paper]{book}
\usepackage{makeidx}
\usepackage{fancyhdr}
\usepackage{graphicx}
\usepackage{multicol}
\usepackage{float}
\usepackage{textcomp}
\usepackage{alltt}
\usepackage{times}
\ifx\pdfoutput\undefined
\usepackage[ps2pdf,
            pagebackref=true,
            colorlinks=true,
            linkcolor=blue
           ]{hyperref}
\usepackage{pspicture}
\else
\usepackage[pdftex,
            pagebackref=true,
            colorlinks=true,
            linkcolor=blue
           ]{hyperref}
\fi
\usepackage[italian]{babel}

\usepackage{doxygen}
\makeindex
\setcounter{tocdepth}{1}
\renewcommand{\footrulewidth}{0.4pt}
\begin{document}
\begin{titlepage}
\vspace*{7cm}
\begin{center}
{\Large Remote Brain Manuale di riferimento\\[1ex]\large 1.0 }\\
\vspace*{1cm}
{\large Generato da Doxygen 1.4.7}\\
\vspace*{0.5cm}
{\small Wed Sep 20 12:28:36 2006}\\
\end{center}
\end{titlepage}
\clearemptydoublepage
\pagenumbering{roman}
\tableofcontents
\clearemptydoublepage
\pagenumbering{arabic}
\chapter{Remote Brain Indice dei namespace}
\section{Remote Brain Lista dei namespace}
Questa \`{e} la lista di tutti i namespace con una loro breve descrizione:\begin{CompactList}
\item\contentsline{section}{\hyperlink{namespacestd}{std} }{\pageref{namespacestd}}{}
\end{CompactList}

\chapter{Remote Brain Indice delle strutture dati}
\section{Remote Brain Strutture dati}
Queste sono le strutture dati con una loro breve descrizione:\begin{CompactList}
\item\contentsline{section}{\hyperlink{classMainWindow}{Main\-Window} }{\pageref{classMainWindow}}{}
\item\contentsline{section}{\hyperlink{classSerialPort}{Serial\-Port} }{\pageref{classSerialPort}}{}
\end{CompactList}

\chapter{Remote Brain Indice dei file}
\section{Remote Brain Lista dei file}
Questa \`{e} una lista di tutti i file con una loro breve descrizione:\begin{CompactList}
\item\contentsline{section}{\hyperlink{main_8cpp}{main.cpp} }{\pageref{main_8cpp}}{}
\item\contentsline{section}{\hyperlink{mainwindow_8cpp}{mainwindow.cpp} }{\pageref{mainwindow_8cpp}}{}
\item\contentsline{section}{\hyperlink{mainwindow_8h}{mainwindow.h} }{\pageref{mainwindow_8h}}{}
\item\contentsline{section}{\hyperlink{SerialPort_8cpp}{Serial\-Port.cpp} }{\pageref{SerialPort_8cpp}}{}
\item\contentsline{section}{\hyperlink{SerialPort_8h}{Serial\-Port.h} }{\pageref{SerialPort_8h}}{}
\end{CompactList}

\chapter{Remote Brain Documentazione dei namespace}
\hypertarget{namespacestd}{
\section{Riferimenti per il namespace std}
\label{namespacestd}\index{std@{std}}
}



\chapter{Remote Brain Documentazione delle classi}
\hypertarget{classMainWindow}{
\section{Riferimenti per la classe Main\-Window}
\label{classMainWindow}\index{MainWindow@{MainWindow}}
}
{\tt \#include $<$mainwindow.h$>$}

\subsection*{Slot pubblici}
\begin{CompactItemize}
\item 
void \hyperlink{classMainWindow_2f955c2c4215bb50acc5d76e5fb99402}{aggiorna} ()
\item 
void \hyperlink{classMainWindow_9155ed247a9ce2d4084737e12be86e7b}{append\-PCInfo} (const char $\ast$info, int i)
\item 
void \hyperlink{classMainWindow_6c1312a8d376cb19366cce1e084ce01b}{append\-PCInfo} (const char $\ast$info, const char $\ast$i)
\item 
void \hyperlink{classMainWindow_1fa8f6b4ad5becd6b6adb85c18f3a238}{append\-PCInfo} (int i)
\item 
void \hyperlink{classMainWindow_00ae5979048fe2b58d6d7f3c526d9f85}{append\-PCInfo} (QString info)
\item 
void \hyperlink{classMainWindow_64f8e53725a778a11501dadf09b8c042}{append\-PCInfo} (char $\ast$info)
\item 
void \hyperlink{classMainWindow_8f842edf62781c42283434f16760a307}{append\-Robot\-Info} (const char $\ast$info, int i)
\item 
void \hyperlink{classMainWindow_74763de0eb539f0af358d83a82588fbb}{append\-Robot\-Info} (const char $\ast$info, const char $\ast$i)
\item 
void \hyperlink{classMainWindow_940ed66f4b3c536f9eee7975099ab015}{append\-Robot\-Info} (int i)
\item 
void \hyperlink{classMainWindow_2f18074a740a58891fd5b989dd71cf3f}{append\-Robot\-Info} (QString info)
\item 
void \hyperlink{classMainWindow_a9b4df136ec15062f2eea2c3673c4b57}{append\-Robot\-Info} (const char $\ast$info)
\end{CompactItemize}
\subsection*{Signal}
\begin{CompactItemize}
\item 
void \hyperlink{classMainWindow_149ee428e4d3b41686c3a5e1879b30a2}{da\-Aggiornare} ()
\end{CompactItemize}
\subsection*{Membri pubblici}
\begin{CompactItemize}
\item 
void \hyperlink{classMainWindow_772ba2e39bd737fa1fb38507c55b6acb}{draw\-Position} ()
\item 
int \hyperlink{classMainWindow_ff0645ac68ad121ee2d70f8f989dca42}{getx} (int \hyperlink{mainwindow_8cpp_ebf8093f61c93230d92191726d5bc846}{robot})
\item 
int \hyperlink{classMainWindow_031e0654d92e39b23d56749657a05d05}{gety} (int \hyperlink{mainwindow_8cpp_ebf8093f61c93230d92191726d5bc846}{robot})
\item 
\hyperlink{classMainWindow_eb676574b78bda84d687fdcfd65d84a6}{Main\-Window} ()
\item 
void \hyperlink{classMainWindow_06a4601d7b3cecfb4432f48e067ac81d}{setx} (int \hyperlink{mainwindow_8cpp_ebf8093f61c93230d92191726d5bc846}{robot}, int newx)
\item 
void \hyperlink{classMainWindow_5ee21538a4cacfcb64a9b504729f2026}{sety} (int \hyperlink{mainwindow_8cpp_ebf8093f61c93230d92191726d5bc846}{robot}, int newy)
\end{CompactItemize}
\subsection*{Campi}
\begin{CompactItemize}
\item 
\hyperlink{classSerialPort}{Serial\-Port} $\ast$ \hyperlink{classMainWindow_730a1b78e42ee50b10b26f34e4bf8b1e}{serialp}
\end{CompactItemize}
\subsection*{Membri protetti}
\begin{CompactItemize}
\item 
void \hyperlink{classMainWindow_59701ebb47f7e2f55977543b6293ba47}{context\-Menu\-Event} (QContext\-Menu\-Event $\ast$event)
\end{CompactItemize}
\subsection*{Slot privati}
\begin{CompactItemize}
\item 
void \hyperlink{classMainWindow_501f096e32ac1de50b0e913803bdd4e8}{close\-Com\-Port} ()
\item 
void \hyperlink{classMainWindow_b20c1e3dbb16cd45ef4f697ce0fb11f7}{goal} ()
\item 
void \hyperlink{classMainWindow_d2272e344e46519f026cd02f419884f1}{mouse\-Press\-Event} (QMouse\-Event $\ast$event)
\item 
void \hyperlink{classMainWindow_faf58be398c427554d4bef465eff4be0}{open\-Com\-Port} ()
\item 
void \hyperlink{classMainWindow_7fea2f4466d9b971fbe4142743a8e3cb}{port\-Sel} ()
\item 
void \hyperlink{classMainWindow_50c5e9556f3dfc6de8e0b064d10aa21f}{robot\-Sel} ()
\item 
void \hyperlink{classMainWindow_0cb6bab0d23b55595337ae6f9e8eba32}{send\-Cmd} ()
\item 
void \hyperlink{classMainWindow_d6073419da050a8657f4adf01acfd2d0}{set\-Pos} ()
\item 
void \hyperlink{classMainWindow_60de64d75454385b23995437f1d72669}{start} ()
\item 
void \hyperlink{classMainWindow_8c528baf37154d347366083f0f816846}{stop} ()
\end{CompactItemize}
\subsection*{Membri privati}
\begin{CompactItemize}
\item 
void \hyperlink{classMainWindow_5176c9496a29e21eacb0f81ca1a29923}{create\-Actions} ()
\item 
void \hyperlink{classMainWindow_5c788b6fcb676c3e1cef7e01eed9a420}{create\-Menus} ()
\end{CompactItemize}
\subsection*{Attributi privati}
\begin{CompactItemize}
\item 
QAction\-Group $\ast$ \hyperlink{classMainWindow_76d93e5045032189c41520235a2196b9}{alignment\-Group}
\item 
QAction $\ast$ \hyperlink{classMainWindow_d618b0cff61434478d762e93add5189e}{exit\-Act}
\item 
QMenu $\ast$ \hyperlink{classMainWindow_db1ab65d1aecac73e58f48557c143b2e}{file\-Menu}
\item 
QLabel $\ast$ \hyperlink{classMainWindow_1778899fb8e16670083da66e055500d0}{info\-Label}
\item 
QPush\-Button $\ast$ \hyperlink{classMainWindow_07ba2d7080a0f0cf1ab683bfe6b80156}{refresh}
\item 
QAction $\ast$ \hyperlink{classMainWindow_168587cb4c00ec6773174b004268191d}{reload\-Act}
\end{CompactItemize}


\subsection{Documentazione dei costruttori e dei distruttori}
\hypertarget{classMainWindow_eb676574b78bda84d687fdcfd65d84a6}{
\index{MainWindow@{Main\-Window}!MainWindow@{MainWindow}}
\index{MainWindow@{MainWindow}!MainWindow@{Main\-Window}}
\subsubsection[MainWindow]{\setlength{\rightskip}{0pt plus 5cm}\hyperlink{classMainWindow}{Main\-Window} ()}}
\label{classMainWindow_eb676574b78bda84d687fdcfd65d84a6}


Main window constructor. 

\subsection{Documentazione delle funzioni membro}
\hypertarget{classMainWindow_2f955c2c4215bb50acc5d76e5fb99402}{
\index{MainWindow@{Main\-Window}!aggiorna@{aggiorna}}
\index{aggiorna@{aggiorna}!MainWindow@{Main\-Window}}
\subsubsection[aggiorna]{\setlength{\rightskip}{0pt plus 5cm}void aggiorna ()\hspace{0.3cm}{\tt  \mbox{[}slot\mbox{]}}}}
\label{classMainWindow_2f955c2c4215bb50acc5d76e5fb99402}


\hypertarget{classMainWindow_9155ed247a9ce2d4084737e12be86e7b}{
\index{MainWindow@{Main\-Window}!appendPCInfo@{appendPCInfo}}
\index{appendPCInfo@{appendPCInfo}!MainWindow@{Main\-Window}}
\subsubsection[appendPCInfo]{\setlength{\rightskip}{0pt plus 5cm}void append\-PCInfo (const char $\ast$ {\em info}, int {\em i})\hspace{0.3cm}{\tt  \mbox{[}slot\mbox{]}}}}
\label{classMainWindow_9155ed247a9ce2d4084737e12be86e7b}


Write a string and an int in PC Info widget. \begin{Desc}
\item[Parametri:]
\begin{description}
\item[{\em info}]The string you have to write. \item[{\em i}]The int you have to write. \end{description}
\end{Desc}
\hypertarget{classMainWindow_6c1312a8d376cb19366cce1e084ce01b}{
\index{MainWindow@{Main\-Window}!appendPCInfo@{appendPCInfo}}
\index{appendPCInfo@{appendPCInfo}!MainWindow@{Main\-Window}}
\subsubsection[appendPCInfo]{\setlength{\rightskip}{0pt plus 5cm}void append\-PCInfo (const char $\ast$ {\em info}, const char $\ast$ {\em i})\hspace{0.3cm}{\tt  \mbox{[}slot\mbox{]}}}}
\label{classMainWindow_6c1312a8d376cb19366cce1e084ce01b}


Write two strings in PC Info widget. \begin{Desc}
\item[Parametri:]
\begin{description}
\item[{\em info}]The first string you have to write. \item[{\em i}]The second string you have to write. \end{description}
\end{Desc}
\hypertarget{classMainWindow_1fa8f6b4ad5becd6b6adb85c18f3a238}{
\index{MainWindow@{Main\-Window}!appendPCInfo@{appendPCInfo}}
\index{appendPCInfo@{appendPCInfo}!MainWindow@{Main\-Window}}
\subsubsection[appendPCInfo]{\setlength{\rightskip}{0pt plus 5cm}void append\-PCInfo (int {\em i})\hspace{0.3cm}{\tt  \mbox{[}slot\mbox{]}}}}
\label{classMainWindow_1fa8f6b4ad5becd6b6adb85c18f3a238}


Write an int in PC Info widget. \begin{Desc}
\item[Parametri:]
\begin{description}
\item[{\em i}]The int you have to write. \end{description}
\end{Desc}
\hypertarget{classMainWindow_00ae5979048fe2b58d6d7f3c526d9f85}{
\index{MainWindow@{Main\-Window}!appendPCInfo@{appendPCInfo}}
\index{appendPCInfo@{appendPCInfo}!MainWindow@{Main\-Window}}
\subsubsection[appendPCInfo]{\setlength{\rightskip}{0pt plus 5cm}void append\-PCInfo (QString {\em info})\hspace{0.3cm}{\tt  \mbox{[}slot\mbox{]}}}}
\label{classMainWindow_00ae5979048fe2b58d6d7f3c526d9f85}


Write a QString in PC Info widget. \begin{Desc}
\item[Parametri:]
\begin{description}
\item[{\em info}]The string you have to write. \end{description}
\end{Desc}
\hypertarget{classMainWindow_64f8e53725a778a11501dadf09b8c042}{
\index{MainWindow@{Main\-Window}!appendPCInfo@{appendPCInfo}}
\index{appendPCInfo@{appendPCInfo}!MainWindow@{Main\-Window}}
\subsubsection[appendPCInfo]{\setlength{\rightskip}{0pt plus 5cm}void append\-PCInfo (char $\ast$ {\em info})\hspace{0.3cm}{\tt  \mbox{[}slot\mbox{]}}}}
\label{classMainWindow_64f8e53725a778a11501dadf09b8c042}


Write a string in PC Info widget. \begin{Desc}
\item[Parametri:]
\begin{description}
\item[{\em info}]The string you have to write. \end{description}
\end{Desc}
\hypertarget{classMainWindow_8f842edf62781c42283434f16760a307}{
\index{MainWindow@{Main\-Window}!appendRobotInfo@{appendRobotInfo}}
\index{appendRobotInfo@{appendRobotInfo}!MainWindow@{Main\-Window}}
\subsubsection[appendRobotInfo]{\setlength{\rightskip}{0pt plus 5cm}void append\-Robot\-Info (const char $\ast$ {\em info}, int {\em i})\hspace{0.3cm}{\tt  \mbox{[}slot\mbox{]}}}}
\label{classMainWindow_8f842edf62781c42283434f16760a307}


Write a string and an int in Robot Info widget. \begin{Desc}
\item[Parametri:]
\begin{description}
\item[{\em info}]The string you have to write. \item[{\em i}]The int you have to write. \end{description}
\end{Desc}
\hypertarget{classMainWindow_74763de0eb539f0af358d83a82588fbb}{
\index{MainWindow@{Main\-Window}!appendRobotInfo@{appendRobotInfo}}
\index{appendRobotInfo@{appendRobotInfo}!MainWindow@{Main\-Window}}
\subsubsection[appendRobotInfo]{\setlength{\rightskip}{0pt plus 5cm}void append\-Robot\-Info (const char $\ast$ {\em info}, const char $\ast$ {\em i})\hspace{0.3cm}{\tt  \mbox{[}slot\mbox{]}}}}
\label{classMainWindow_74763de0eb539f0af358d83a82588fbb}


Write two strings in Robot Info widget. \begin{Desc}
\item[Parametri:]
\begin{description}
\item[{\em info}]The first string you have to write. \item[{\em i}]The second string you have to write. \end{description}
\end{Desc}
\hypertarget{classMainWindow_940ed66f4b3c536f9eee7975099ab015}{
\index{MainWindow@{Main\-Window}!appendRobotInfo@{appendRobotInfo}}
\index{appendRobotInfo@{appendRobotInfo}!MainWindow@{Main\-Window}}
\subsubsection[appendRobotInfo]{\setlength{\rightskip}{0pt plus 5cm}void append\-Robot\-Info (int {\em i})\hspace{0.3cm}{\tt  \mbox{[}slot\mbox{]}}}}
\label{classMainWindow_940ed66f4b3c536f9eee7975099ab015}


Write an int in Robot Info widget. \begin{Desc}
\item[Parametri:]
\begin{description}
\item[{\em i}]The int you have to write. \end{description}
\end{Desc}
\hypertarget{classMainWindow_2f18074a740a58891fd5b989dd71cf3f}{
\index{MainWindow@{Main\-Window}!appendRobotInfo@{appendRobotInfo}}
\index{appendRobotInfo@{appendRobotInfo}!MainWindow@{Main\-Window}}
\subsubsection[appendRobotInfo]{\setlength{\rightskip}{0pt plus 5cm}void append\-Robot\-Info (QString {\em info})\hspace{0.3cm}{\tt  \mbox{[}slot\mbox{]}}}}
\label{classMainWindow_2f18074a740a58891fd5b989dd71cf3f}


Write a QString in Robot Info widget. \begin{Desc}
\item[Parametri:]
\begin{description}
\item[{\em info}]The string you have to write. \end{description}
\end{Desc}
\hypertarget{classMainWindow_a9b4df136ec15062f2eea2c3673c4b57}{
\index{MainWindow@{Main\-Window}!appendRobotInfo@{appendRobotInfo}}
\index{appendRobotInfo@{appendRobotInfo}!MainWindow@{Main\-Window}}
\subsubsection[appendRobotInfo]{\setlength{\rightskip}{0pt plus 5cm}void append\-Robot\-Info (const char $\ast$ {\em info})\hspace{0.3cm}{\tt  \mbox{[}slot\mbox{]}}}}
\label{classMainWindow_a9b4df136ec15062f2eea2c3673c4b57}


Write a string in Robot Info widget. \begin{Desc}
\item[Parametri:]
\begin{description}
\item[{\em info}]The string you have to write. \end{description}
\end{Desc}
\hypertarget{classMainWindow_501f096e32ac1de50b0e913803bdd4e8}{
\index{MainWindow@{Main\-Window}!closeComPort@{closeComPort}}
\index{closeComPort@{closeComPort}!MainWindow@{Main\-Window}}
\subsubsection[closeComPort]{\setlength{\rightskip}{0pt plus 5cm}void close\-Com\-Port ()\hspace{0.3cm}{\tt  \mbox{[}private, slot\mbox{]}}}}
\label{classMainWindow_501f096e32ac1de50b0e913803bdd4e8}


Method associated with Close Com\-Port button Close the handle of the open port and terminate the thread \hypertarget{classMainWindow_59701ebb47f7e2f55977543b6293ba47}{
\index{MainWindow@{Main\-Window}!contextMenuEvent@{contextMenuEvent}}
\index{contextMenuEvent@{contextMenuEvent}!MainWindow@{Main\-Window}}
\subsubsection[contextMenuEvent]{\setlength{\rightskip}{0pt plus 5cm}void context\-Menu\-Event (QContext\-Menu\-Event $\ast$ {\em event})\hspace{0.3cm}{\tt  \mbox{[}protected\mbox{]}}}}
\label{classMainWindow_59701ebb47f7e2f55977543b6293ba47}


Event associated with context menu \hypertarget{classMainWindow_5176c9496a29e21eacb0f81ca1a29923}{
\index{MainWindow@{Main\-Window}!createActions@{createActions}}
\index{createActions@{createActions}!MainWindow@{Main\-Window}}
\subsubsection[createActions]{\setlength{\rightskip}{0pt plus 5cm}void create\-Actions ()\hspace{0.3cm}{\tt  \mbox{[}private\mbox{]}}}}
\label{classMainWindow_5176c9496a29e21eacb0f81ca1a29923}


Create menu actions \hypertarget{classMainWindow_5c788b6fcb676c3e1cef7e01eed9a420}{
\index{MainWindow@{Main\-Window}!createMenus@{createMenus}}
\index{createMenus@{createMenus}!MainWindow@{Main\-Window}}
\subsubsection[createMenus]{\setlength{\rightskip}{0pt plus 5cm}void create\-Menus ()\hspace{0.3cm}{\tt  \mbox{[}private\mbox{]}}}}
\label{classMainWindow_5c788b6fcb676c3e1cef7e01eed9a420}


Create menus \hypertarget{classMainWindow_149ee428e4d3b41686c3a5e1879b30a2}{
\index{MainWindow@{Main\-Window}!daAggiornare@{daAggiornare}}
\index{daAggiornare@{daAggiornare}!MainWindow@{Main\-Window}}
\subsubsection[daAggiornare]{\setlength{\rightskip}{0pt plus 5cm}void da\-Aggiornare ()\hspace{0.3cm}{\tt  \mbox{[}signal\mbox{]}}}}
\label{classMainWindow_149ee428e4d3b41686c3a5e1879b30a2}


\hypertarget{classMainWindow_772ba2e39bd737fa1fb38507c55b6acb}{
\index{MainWindow@{Main\-Window}!drawPosition@{drawPosition}}
\index{drawPosition@{drawPosition}!MainWindow@{Main\-Window}}
\subsubsection[drawPosition]{\setlength{\rightskip}{0pt plus 5cm}void draw\-Position ()}}
\label{classMainWindow_772ba2e39bd737fa1fb38507c55b6acb}


Draw a cross on the field image. Each cross indicates a Robot's position. \hypertarget{classMainWindow_ff0645ac68ad121ee2d70f8f989dca42}{
\index{MainWindow@{Main\-Window}!getx@{getx}}
\index{getx@{getx}!MainWindow@{Main\-Window}}
\subsubsection[getx]{\setlength{\rightskip}{0pt plus 5cm}int getx (int {\em robot})}}
\label{classMainWindow_ff0645ac68ad121ee2d70f8f989dca42}


Get the x coordinate of a Robot. \begin{Desc}
\item[Parametri:]
\begin{description}
\item[{\em robot}]The Robot's number. \end{description}
\end{Desc}
\hypertarget{classMainWindow_031e0654d92e39b23d56749657a05d05}{
\index{MainWindow@{Main\-Window}!gety@{gety}}
\index{gety@{gety}!MainWindow@{Main\-Window}}
\subsubsection[gety]{\setlength{\rightskip}{0pt plus 5cm}int gety (int {\em robot})}}
\label{classMainWindow_031e0654d92e39b23d56749657a05d05}


Get the y coordinate of a Robot. \begin{Desc}
\item[Parametri:]
\begin{description}
\item[{\em robot}]The Robot's number. \end{description}
\end{Desc}
\hypertarget{classMainWindow_b20c1e3dbb16cd45ef4f697ce0fb11f7}{
\index{MainWindow@{Main\-Window}!goal@{goal}}
\index{goal@{goal}!MainWindow@{Main\-Window}}
\subsubsection[goal]{\setlength{\rightskip}{0pt plus 5cm}void goal ()\hspace{0.3cm}{\tt  \mbox{[}private, slot\mbox{]}}}}
\label{classMainWindow_b20c1e3dbb16cd45ef4f697ce0fb11f7}


Method associated with Goal button. Sends the following command: Ri\-GOAL=y or Ri\-GOAL=b where i is the robot's number y is the yellow goal b is the blue goal \hypertarget{classMainWindow_d2272e344e46519f026cd02f419884f1}{
\index{MainWindow@{Main\-Window}!mousePressEvent@{mousePressEvent}}
\index{mousePressEvent@{mousePressEvent}!MainWindow@{Main\-Window}}
\subsubsection[mousePressEvent]{\setlength{\rightskip}{0pt plus 5cm}void mouse\-Press\-Event (QMouse\-Event $\ast$ {\em event})\hspace{0.3cm}{\tt  \mbox{[}private, slot\mbox{]}}}}
\label{classMainWindow_d2272e344e46519f026cd02f419884f1}


Capture a mouse event in the main window \hypertarget{classMainWindow_faf58be398c427554d4bef465eff4be0}{
\index{MainWindow@{Main\-Window}!openComPort@{openComPort}}
\index{openComPort@{openComPort}!MainWindow@{Main\-Window}}
\subsubsection[openComPort]{\setlength{\rightskip}{0pt plus 5cm}void open\-Com\-Port ()\hspace{0.3cm}{\tt  \mbox{[}private, slot\mbox{]}}}}
\label{classMainWindow_faf58be398c427554d4bef465eff4be0}


Method associated with the Open\-Port button. Open a connection on the selected COM port. If the connection is created successfully, a thread is started waiting for incoming strings. \hypertarget{classMainWindow_7fea2f4466d9b971fbe4142743a8e3cb}{
\index{MainWindow@{Main\-Window}!portSel@{portSel}}
\index{portSel@{portSel}!MainWindow@{Main\-Window}}
\subsubsection[portSel]{\setlength{\rightskip}{0pt plus 5cm}void port\-Sel ()\hspace{0.3cm}{\tt  \mbox{[}private, slot\mbox{]}}}}
\label{classMainWindow_7fea2f4466d9b971fbe4142743a8e3cb}


Method associated with COM port selection combo \hypertarget{classMainWindow_50c5e9556f3dfc6de8e0b064d10aa21f}{
\index{MainWindow@{Main\-Window}!robotSel@{robotSel}}
\index{robotSel@{robotSel}!MainWindow@{Main\-Window}}
\subsubsection[robotSel]{\setlength{\rightskip}{0pt plus 5cm}void robot\-Sel ()\hspace{0.3cm}{\tt  \mbox{[}private, slot\mbox{]}}}}
\label{classMainWindow_50c5e9556f3dfc6de8e0b064d10aa21f}


Method associated with robot selection combo \hypertarget{classMainWindow_0cb6bab0d23b55595337ae6f9e8eba32}{
\index{MainWindow@{Main\-Window}!sendCmd@{sendCmd}}
\index{sendCmd@{sendCmd}!MainWindow@{Main\-Window}}
\subsubsection[sendCmd]{\setlength{\rightskip}{0pt plus 5cm}void send\-Cmd ()\hspace{0.3cm}{\tt  \mbox{[}private, slot\mbox{]}}}}
\label{classMainWindow_0cb6bab0d23b55595337ae6f9e8eba32}


Method associated with Send Command button. Sends the string specified in the associated text box. \hypertarget{classMainWindow_d6073419da050a8657f4adf01acfd2d0}{
\index{MainWindow@{Main\-Window}!setPos@{setPos}}
\index{setPos@{setPos}!MainWindow@{Main\-Window}}
\subsubsection[setPos]{\setlength{\rightskip}{0pt plus 5cm}void set\-Pos ()\hspace{0.3cm}{\tt  \mbox{[}private, slot\mbox{]}}}}
\label{classMainWindow_d6073419da050a8657f4adf01acfd2d0}


Method associated with Set Posistion button. Send the following command: Ri\-SETPOS.X=x.Y=y.T=t where i is the robot's number x,y the position t the orientation. \hypertarget{classMainWindow_06a4601d7b3cecfb4432f48e067ac81d}{
\index{MainWindow@{Main\-Window}!setx@{setx}}
\index{setx@{setx}!MainWindow@{Main\-Window}}
\subsubsection[setx]{\setlength{\rightskip}{0pt plus 5cm}void setx (int {\em robot}, int {\em newx})}}
\label{classMainWindow_06a4601d7b3cecfb4432f48e067ac81d}


Set the x coordinate of a Robot. \begin{Desc}
\item[Parametri:]
\begin{description}
\item[{\em robot}]The Robot's number. \item[{\em newx}]The x coordinate. \end{description}
\end{Desc}
\hypertarget{classMainWindow_5ee21538a4cacfcb64a9b504729f2026}{
\index{MainWindow@{Main\-Window}!sety@{sety}}
\index{sety@{sety}!MainWindow@{Main\-Window}}
\subsubsection[sety]{\setlength{\rightskip}{0pt plus 5cm}void sety (int {\em robot}, int {\em newy})}}
\label{classMainWindow_5ee21538a4cacfcb64a9b504729f2026}


Set the y coordinate of a Robot. \begin{Desc}
\item[Parametri:]
\begin{description}
\item[{\em robot}]The Robot's number. \item[{\em newy}]The y coordinate. \end{description}
\end{Desc}
\hypertarget{classMainWindow_60de64d75454385b23995437f1d72669}{
\index{MainWindow@{Main\-Window}!start@{start}}
\index{start@{start}!MainWindow@{Main\-Window}}
\subsubsection[start]{\setlength{\rightskip}{0pt plus 5cm}void start ()\hspace{0.3cm}{\tt  \mbox{[}private, slot\mbox{]}}}}
\label{classMainWindow_60de64d75454385b23995437f1d72669}


Method associated with Start button. Sends the following command: Ri\-START where i is the robot's number. \hypertarget{classMainWindow_8c528baf37154d347366083f0f816846}{
\index{MainWindow@{Main\-Window}!stop@{stop}}
\index{stop@{stop}!MainWindow@{Main\-Window}}
\subsubsection[stop]{\setlength{\rightskip}{0pt plus 5cm}void stop ()\hspace{0.3cm}{\tt  \mbox{[}private, slot\mbox{]}}}}
\label{classMainWindow_8c528baf37154d347366083f0f816846}


Method associated with Stop button. Sends the following command: Ri\-STOP where i is the robot's number. 

\subsection{Documentazione dei campi}
\hypertarget{classMainWindow_76d93e5045032189c41520235a2196b9}{
\index{MainWindow@{Main\-Window}!alignmentGroup@{alignmentGroup}}
\index{alignmentGroup@{alignmentGroup}!MainWindow@{Main\-Window}}
\subsubsection[alignmentGroup]{\setlength{\rightskip}{0pt plus 5cm}QAction\-Group$\ast$ \hyperlink{classMainWindow_76d93e5045032189c41520235a2196b9}{alignment\-Group}\hspace{0.3cm}{\tt  \mbox{[}private\mbox{]}}}}
\label{classMainWindow_76d93e5045032189c41520235a2196b9}


\hypertarget{classMainWindow_d618b0cff61434478d762e93add5189e}{
\index{MainWindow@{Main\-Window}!exitAct@{exitAct}}
\index{exitAct@{exitAct}!MainWindow@{Main\-Window}}
\subsubsection[exitAct]{\setlength{\rightskip}{0pt plus 5cm}QAction$\ast$ \hyperlink{classMainWindow_d618b0cff61434478d762e93add5189e}{exit\-Act}\hspace{0.3cm}{\tt  \mbox{[}private\mbox{]}}}}
\label{classMainWindow_d618b0cff61434478d762e93add5189e}


\hypertarget{classMainWindow_db1ab65d1aecac73e58f48557c143b2e}{
\index{MainWindow@{Main\-Window}!fileMenu@{fileMenu}}
\index{fileMenu@{fileMenu}!MainWindow@{Main\-Window}}
\subsubsection[fileMenu]{\setlength{\rightskip}{0pt plus 5cm}QMenu$\ast$ \hyperlink{classMainWindow_db1ab65d1aecac73e58f48557c143b2e}{file\-Menu}\hspace{0.3cm}{\tt  \mbox{[}private\mbox{]}}}}
\label{classMainWindow_db1ab65d1aecac73e58f48557c143b2e}


\hypertarget{classMainWindow_1778899fb8e16670083da66e055500d0}{
\index{MainWindow@{Main\-Window}!infoLabel@{infoLabel}}
\index{infoLabel@{infoLabel}!MainWindow@{Main\-Window}}
\subsubsection[infoLabel]{\setlength{\rightskip}{0pt plus 5cm}QLabel$\ast$ \hyperlink{classMainWindow_1778899fb8e16670083da66e055500d0}{info\-Label}\hspace{0.3cm}{\tt  \mbox{[}private\mbox{]}}}}
\label{classMainWindow_1778899fb8e16670083da66e055500d0}


\hypertarget{classMainWindow_07ba2d7080a0f0cf1ab683bfe6b80156}{
\index{MainWindow@{Main\-Window}!refresh@{refresh}}
\index{refresh@{refresh}!MainWindow@{Main\-Window}}
\subsubsection[refresh]{\setlength{\rightskip}{0pt plus 5cm}QPush\-Button$\ast$ \hyperlink{classMainWindow_07ba2d7080a0f0cf1ab683bfe6b80156}{refresh}\hspace{0.3cm}{\tt  \mbox{[}private\mbox{]}}}}
\label{classMainWindow_07ba2d7080a0f0cf1ab683bfe6b80156}


\hypertarget{classMainWindow_168587cb4c00ec6773174b004268191d}{
\index{MainWindow@{Main\-Window}!reloadAct@{reloadAct}}
\index{reloadAct@{reloadAct}!MainWindow@{Main\-Window}}
\subsubsection[reloadAct]{\setlength{\rightskip}{0pt plus 5cm}QAction$\ast$ \hyperlink{classMainWindow_168587cb4c00ec6773174b004268191d}{reload\-Act}\hspace{0.3cm}{\tt  \mbox{[}private\mbox{]}}}}
\label{classMainWindow_168587cb4c00ec6773174b004268191d}


\hypertarget{classMainWindow_730a1b78e42ee50b10b26f34e4bf8b1e}{
\index{MainWindow@{Main\-Window}!serialp@{serialp}}
\index{serialp@{serialp}!MainWindow@{Main\-Window}}
\subsubsection[serialp]{\setlength{\rightskip}{0pt plus 5cm}\hyperlink{classSerialPort}{Serial\-Port}$\ast$ \hyperlink{classMainWindow_730a1b78e42ee50b10b26f34e4bf8b1e}{serialp}}}
\label{classMainWindow_730a1b78e42ee50b10b26f34e4bf8b1e}




La documentazione per questa classe \`{e} stata generata a partire dai seguenti file:\begin{CompactItemize}
\item 
\hyperlink{mainwindow_8h}{mainwindow.h}\item 
\hyperlink{mainwindow_8cpp}{mainwindow.cpp}\end{CompactItemize}

\hypertarget{classSerialPort}{
\section{Riferimenti per la classe Serial\-Port}
\label{classSerialPort}\index{SerialPort@{SerialPort}}
}
{\tt \#include $<$Serial\-Port.h$>$}

\subsection*{Membri pubblici}
\begin{CompactItemize}
\item 
void \hyperlink{classSerialPort_13a43e6d814de94978c515cb084873b1}{run} ()
\item 
\hyperlink{classSerialPort_e759c132d52bdeaddba3ebd737c77821}{Serial\-Port} (\hyperlink{classMainWindow}{Main\-Window} $\ast$m\-W)
\item 
\hyperlink{classSerialPort_0f67eac24b774429a205b00237f12039}{$\sim$Serial\-Port} ()
\end{CompactItemize}


\subsection{Documentazione dei costruttori e dei distruttori}
\hypertarget{classSerialPort_e759c132d52bdeaddba3ebd737c77821}{
\index{SerialPort@{Serial\-Port}!SerialPort@{SerialPort}}
\index{SerialPort@{SerialPort}!SerialPort@{Serial\-Port}}
\subsubsection[SerialPort]{\setlength{\rightskip}{0pt plus 5cm}\hyperlink{classSerialPort}{Serial\-Port} (\hyperlink{classMainWindow}{Main\-Window} $\ast$ {\em m\-W})}}
\label{classSerialPort_e759c132d52bdeaddba3ebd737c77821}


Constructor \hypertarget{classSerialPort_0f67eac24b774429a205b00237f12039}{
\index{SerialPort@{Serial\-Port}!~SerialPort@{$\sim$SerialPort}}
\index{~SerialPort@{$\sim$SerialPort}!SerialPort@{Serial\-Port}}
\subsubsection[$\sim$SerialPort]{\setlength{\rightskip}{0pt plus 5cm}$\sim$\hyperlink{classSerialPort}{Serial\-Port} ()}}
\label{classSerialPort_0f67eac24b774429a205b00237f12039}


Destructor 

\subsection{Documentazione delle funzioni membro}
\hypertarget{classSerialPort_13a43e6d814de94978c515cb084873b1}{
\index{SerialPort@{Serial\-Port}!run@{run}}
\index{run@{run}!SerialPort@{Serial\-Port}}
\subsubsection[run]{\setlength{\rightskip}{0pt plus 5cm}void run ()}}
\label{classSerialPort_13a43e6d814de94978c515cb084873b1}


Thread loop 

La documentazione per questa classe \`{e} stata generata a partire dai seguenti file:\begin{CompactItemize}
\item 
\hyperlink{SerialPort_8h}{Serial\-Port.h}\item 
\hyperlink{SerialPort_8cpp}{Serial\-Port.cpp}\end{CompactItemize}

\chapter{Remote Brain Documentazione dei file}
\hypertarget{main_8cpp}{
\section{Riferimenti per il file main.cpp}
\label{main_8cpp}\index{main.cpp@{main.cpp}}
}
{\tt \#include $<$QApplication$>$}\par
{\tt \#include \char`\"{}Serial\-Port.h\char`\"{}}\par
{\tt \#include \char`\"{}mainwindow.h\char`\"{}}\par
\subsection*{Funzioni}
\begin{CompactItemize}
\item 
int \hyperlink{main_8cpp_0ddf1224851353fc92bfbff6f499fa97}{main} (int argc, char $\ast$argv\mbox{[}$\,$\mbox{]})
\end{CompactItemize}


\subsection{Documentazione delle funzioni}
\hypertarget{main_8cpp_0ddf1224851353fc92bfbff6f499fa97}{
\index{main.cpp@{main.cpp}!main@{main}}
\index{main@{main}!main.cpp@{main.cpp}}
\subsubsection[main]{\setlength{\rightskip}{0pt plus 5cm}int main (int {\em argc}, char $\ast$ {\em argv}\mbox{[}$\,$\mbox{]})}}
\label{main_8cpp_0ddf1224851353fc92bfbff6f499fa97}



\hypertarget{mainwindow_8cpp}{
\section{Riferimenti per il file mainwindow.cpp}
\label{mainwindow_8cpp}\index{mainwindow.cpp@{mainwindow.cpp}}
}
{\tt \#include $<$Qt\-Gui$>$}\par
{\tt \#include $<$QPoint$>$}\par
{\tt \#include $<$QPainter$>$}\par
{\tt \#include \char`\"{}mainwindow.h\char`\"{}}\par
\subsection*{Namespace}
\begin{CompactItemize}
\item 
namespace \hyperlink{namespacestd}{std}
\end{CompactItemize}
\subsection*{Definizioni}
\begin{CompactItemize}
\item 
\#define \hyperlink{mainwindow_8cpp_be0f0011267471da0f8ebcf226dd05f8}{FIELDH}~419
\item 
\#define \hyperlink{mainwindow_8cpp_2f469c925fcf21dd6d918a036bb3f4e3}{FIELDW}~570
\item 
\#define \hyperlink{mainwindow_8cpp_696ef618a97a7e6909fef5278343cf43}{NUMBEROFROBOTS}~2
\item 
\#define \hyperlink{mainwindow_8cpp_515119526a2e89ab018116ac95b628c0}{RECVBUFFERSIZE}~512
\end{CompactItemize}
\subsection*{Funzioni}
\begin{CompactItemize}
\item 
void \hyperlink{mainwindow_8cpp_69a50a89c86348f01cc23a429685e9dd}{define\-Robot\-Colors} ()
\item 
DWORD WINAPI \hyperlink{mainwindow_8cpp_3ec429244ead8f4eb9f75e05fbfee35e}{funzione} (LPDWORD $\ast$lpdw\-Param)
\item 
void \hyperlink{mainwindow_8cpp_046465f393d0876a74f11c66c1c95fa4}{listeningthread} ()
\end{CompactItemize}
\subsection*{Variabili}
\begin{CompactItemize}
\item 
int \hyperlink{mainwindow_8cpp_61ee361cdeed1340ff62da8e9f2d7b46}{b} \mbox{[}NUMBEROFROBOTS\mbox{]}
\item 
QPush\-Button $\ast$ \hyperlink{mainwindow_8cpp_614d16ffffa52eb68de5bd7ba55b4cff}{close\-Port\-Button}
\item 
QText\-Edit $\ast$ \hyperlink{mainwindow_8cpp_560cbdfe5d8d9cf91c8065f38d391ba4}{cmd\-Line}
\item 
QCombo\-Box $\ast$ \hyperlink{mainwindow_8cpp_bb2810052111e8cc62004fc5e3d2858f}{combo}
\item 
QCombo\-Box $\ast$ \hyperlink{mainwindow_8cpp_9c2b7044142a72d814ecff31f02fab48}{combo\-Field\-Size}
\item 
QCombo\-Box $\ast$ \hyperlink{mainwindow_8cpp_f3706bd1004df2f034b2d4248aa005c1}{combo\-Port}
\item 
char $\ast$ \hyperlink{mainwindow_8cpp_85b548c0a2f5b990fd45b1846f702bed}{fieldport} = \char`\"{}y\char`\"{}
\item 
int \hyperlink{mainwindow_8cpp_2e153d156330e018184991dd02e75009}{g} \mbox{[}NUMBEROFROBOTS\mbox{]}
\item 
QPush\-Button $\ast$ \hyperlink{mainwindow_8cpp_b6a9976891cb20971afedcb268e596a0}{go\-Button}
\item 
QLabel $\ast$ \hyperlink{mainwindow_8cpp_5b3043aacdbae62352f5993a33c4a4b7}{image\-Label}
\item 
QText\-Edit $\ast$ \hyperlink{mainwindow_8cpp_3fbb8173a9510be37c564f103a543aaa}{info1}
\item 
QText\-Edit $\ast$ \hyperlink{mainwindow_8cpp_850f93f80a2433d5fadf30d5e5e866f4}{info2}
\item 
char $\ast$ \hyperlink{mainwindow_8cpp_0b2e8c7f76df48129f994ecc46d5c66c}{message}
\item 
QPush\-Button $\ast$ \hyperlink{mainwindow_8cpp_84351fcfbbed5a81cce783cb68e2eaf6}{open\-Port\-Button}
\item 
char $\ast$ \hyperlink{mainwindow_8cpp_00eeca5cdf915ad3d1fd8dbf28ab156d}{portacom}
\item 
HANDLE \hyperlink{mainwindow_8cpp_6b10b2ae96c856016dc4e92021ea50af}{port\-Handle}
\item 
int \hyperlink{mainwindow_8cpp_348a8673070b1d1e2f91feb1149cb76b}{pos\_\-x} \mbox{[}NUMBEROFROBOTS\mbox{]}
\item 
int \hyperlink{mainwindow_8cpp_f803113f8db598bbb5cc7cb55c53fd72}{pos\_\-y} \mbox{[}NUMBEROFROBOTS\mbox{]}
\item 
int \hyperlink{mainwindow_8cpp_a194f1a2c336647d7165309ff921f576}{r} \mbox{[}NUMBEROFROBOTS\mbox{]}
\item 
char \hyperlink{mainwindow_8cpp_65bf9d32222c82b489f99fcb4f892edd}{recvbuffer} \mbox{[}RECVBUFFERSIZE\mbox{]}
\item 
int \hyperlink{mainwindow_8cpp_ebf8093f61c93230d92191726d5bc846}{robot}
\item 
bool \hyperlink{mainwindow_8cpp_e31aae5c36d2da2116798bb42bf14d73}{robotchanged}
\item 
QPush\-Button $\ast$ \hyperlink{mainwindow_8cpp_852e2307ac8c879169cd0f76ed135c1b}{set\-Pos\-Button}
\item 
QPush\-Button $\ast$ \hyperlink{mainwindow_8cpp_a68dc4dea54cd443bda2a53f9b0621cd}{start\-Button}
\item 
QPush\-Button $\ast$ \hyperlink{mainwindow_8cpp_12a122b59e7d02ed91703cb0568faa3c}{stop\-Button}
\item 
QString \hyperlink{mainwindow_8cpp_288b834f8b483958be715bc77558bf46}{str}
\end{CompactItemize}


\subsection{Documentazione delle definizioni}
\hypertarget{mainwindow_8cpp_be0f0011267471da0f8ebcf226dd05f8}{
\index{mainwindow.cpp@{mainwindow.cpp}!FIELDH@{FIELDH}}
\index{FIELDH@{FIELDH}!mainwindow.cpp@{mainwindow.cpp}}
\subsubsection[FIELDH]{\setlength{\rightskip}{0pt plus 5cm}\#define FIELDH~419}}
\label{mainwindow_8cpp_be0f0011267471da0f8ebcf226dd05f8}


\hypertarget{mainwindow_8cpp_2f469c925fcf21dd6d918a036bb3f4e3}{
\index{mainwindow.cpp@{mainwindow.cpp}!FIELDW@{FIELDW}}
\index{FIELDW@{FIELDW}!mainwindow.cpp@{mainwindow.cpp}}
\subsubsection[FIELDW]{\setlength{\rightskip}{0pt plus 5cm}\#define FIELDW~570}}
\label{mainwindow_8cpp_2f469c925fcf21dd6d918a036bb3f4e3}


Title: Main\-Window.cpp 

Description: Implementation of the graphical interface of the Remote Brain application. \begin{Desc}
\item[Autore:]Andrea Maglie (Matr.456188) \end{Desc}
\begin{Desc}
\item[Versione:]1.0 \end{Desc}
\hypertarget{mainwindow_8cpp_696ef618a97a7e6909fef5278343cf43}{
\index{mainwindow.cpp@{mainwindow.cpp}!NUMBEROFROBOTS@{NUMBEROFROBOTS}}
\index{NUMBEROFROBOTS@{NUMBEROFROBOTS}!mainwindow.cpp@{mainwindow.cpp}}
\subsubsection[NUMBEROFROBOTS]{\setlength{\rightskip}{0pt plus 5cm}\#define NUMBEROFROBOTS~2}}
\label{mainwindow_8cpp_696ef618a97a7e6909fef5278343cf43}


\hypertarget{mainwindow_8cpp_515119526a2e89ab018116ac95b628c0}{
\index{mainwindow.cpp@{mainwindow.cpp}!RECVBUFFERSIZE@{RECVBUFFERSIZE}}
\index{RECVBUFFERSIZE@{RECVBUFFERSIZE}!mainwindow.cpp@{mainwindow.cpp}}
\subsubsection[RECVBUFFERSIZE]{\setlength{\rightskip}{0pt plus 5cm}\#define RECVBUFFERSIZE~512}}
\label{mainwindow_8cpp_515119526a2e89ab018116ac95b628c0}




\subsection{Documentazione delle funzioni}
\hypertarget{mainwindow_8cpp_69a50a89c86348f01cc23a429685e9dd}{
\index{mainwindow.cpp@{mainwindow.cpp}!defineRobotColors@{defineRobotColors}}
\index{defineRobotColors@{defineRobotColors}!mainwindow.cpp@{mainwindow.cpp}}
\subsubsection[defineRobotColors]{\setlength{\rightskip}{0pt plus 5cm}void define\-Robot\-Colors ()}}
\label{mainwindow_8cpp_69a50a89c86348f01cc23a429685e9dd}


Customize the robot indicator's color. \hypertarget{mainwindow_8cpp_3ec429244ead8f4eb9f75e05fbfee35e}{
\index{mainwindow.cpp@{mainwindow.cpp}!funzione@{funzione}}
\index{funzione@{funzione}!mainwindow.cpp@{mainwindow.cpp}}
\subsubsection[funzione]{\setlength{\rightskip}{0pt plus 5cm}DWORD WINAPI funzione (LPDWORD $\ast$ {\em lpdw\-Param})}}
\label{mainwindow_8cpp_3ec429244ead8f4eb9f75e05fbfee35e}


\hypertarget{mainwindow_8cpp_046465f393d0876a74f11c66c1c95fa4}{
\index{mainwindow.cpp@{mainwindow.cpp}!listeningthread@{listeningthread}}
\index{listeningthread@{listeningthread}!mainwindow.cpp@{mainwindow.cpp}}
\subsubsection[listeningthread]{\setlength{\rightskip}{0pt plus 5cm}void listeningthread ()}}
\label{mainwindow_8cpp_046465f393d0876a74f11c66c1c95fa4}




\subsection{Documentazione delle variabili}
\hypertarget{mainwindow_8cpp_61ee361cdeed1340ff62da8e9f2d7b46}{
\index{mainwindow.cpp@{mainwindow.cpp}!b@{b}}
\index{b@{b}!mainwindow.cpp@{mainwindow.cpp}}
\subsubsection[b]{\setlength{\rightskip}{0pt plus 5cm}int \hyperlink{mainwindow_8cpp_61ee361cdeed1340ff62da8e9f2d7b46}{b}\mbox{[}NUMBEROFROBOTS\mbox{]}}}
\label{mainwindow_8cpp_61ee361cdeed1340ff62da8e9f2d7b46}


\hypertarget{mainwindow_8cpp_614d16ffffa52eb68de5bd7ba55b4cff}{
\index{mainwindow.cpp@{mainwindow.cpp}!closePortButton@{closePortButton}}
\index{closePortButton@{closePortButton}!mainwindow.cpp@{mainwindow.cpp}}
\subsubsection[closePortButton]{\setlength{\rightskip}{0pt plus 5cm}QPush\-Button$\ast$ \hyperlink{mainwindow_8cpp_614d16ffffa52eb68de5bd7ba55b4cff}{close\-Port\-Button}}}
\label{mainwindow_8cpp_614d16ffffa52eb68de5bd7ba55b4cff}


\hypertarget{mainwindow_8cpp_560cbdfe5d8d9cf91c8065f38d391ba4}{
\index{mainwindow.cpp@{mainwindow.cpp}!cmdLine@{cmdLine}}
\index{cmdLine@{cmdLine}!mainwindow.cpp@{mainwindow.cpp}}
\subsubsection[cmdLine]{\setlength{\rightskip}{0pt plus 5cm}QText\-Edit$\ast$ \hyperlink{mainwindow_8cpp_560cbdfe5d8d9cf91c8065f38d391ba4}{cmd\-Line}}}
\label{mainwindow_8cpp_560cbdfe5d8d9cf91c8065f38d391ba4}


\hypertarget{mainwindow_8cpp_bb2810052111e8cc62004fc5e3d2858f}{
\index{mainwindow.cpp@{mainwindow.cpp}!combo@{combo}}
\index{combo@{combo}!mainwindow.cpp@{mainwindow.cpp}}
\subsubsection[combo]{\setlength{\rightskip}{0pt plus 5cm}QCombo\-Box$\ast$ \hyperlink{mainwindow_8cpp_bb2810052111e8cc62004fc5e3d2858f}{combo}}}
\label{mainwindow_8cpp_bb2810052111e8cc62004fc5e3d2858f}


\hypertarget{mainwindow_8cpp_9c2b7044142a72d814ecff31f02fab48}{
\index{mainwindow.cpp@{mainwindow.cpp}!comboFieldSize@{comboFieldSize}}
\index{comboFieldSize@{comboFieldSize}!mainwindow.cpp@{mainwindow.cpp}}
\subsubsection[comboFieldSize]{\setlength{\rightskip}{0pt plus 5cm}QCombo\-Box$\ast$ \hyperlink{mainwindow_8cpp_9c2b7044142a72d814ecff31f02fab48}{combo\-Field\-Size}}}
\label{mainwindow_8cpp_9c2b7044142a72d814ecff31f02fab48}


\hypertarget{mainwindow_8cpp_f3706bd1004df2f034b2d4248aa005c1}{
\index{mainwindow.cpp@{mainwindow.cpp}!comboPort@{comboPort}}
\index{comboPort@{comboPort}!mainwindow.cpp@{mainwindow.cpp}}
\subsubsection[comboPort]{\setlength{\rightskip}{0pt plus 5cm}QCombo\-Box$\ast$ \hyperlink{mainwindow_8cpp_f3706bd1004df2f034b2d4248aa005c1}{combo\-Port}}}
\label{mainwindow_8cpp_f3706bd1004df2f034b2d4248aa005c1}


\hypertarget{mainwindow_8cpp_85b548c0a2f5b990fd45b1846f702bed}{
\index{mainwindow.cpp@{mainwindow.cpp}!fieldport@{fieldport}}
\index{fieldport@{fieldport}!mainwindow.cpp@{mainwindow.cpp}}
\subsubsection[fieldport]{\setlength{\rightskip}{0pt plus 5cm}char$\ast$ \hyperlink{mainwindow_8cpp_85b548c0a2f5b990fd45b1846f702bed}{fieldport} = \char`\"{}y\char`\"{}}}
\label{mainwindow_8cpp_85b548c0a2f5b990fd45b1846f702bed}


\hypertarget{mainwindow_8cpp_2e153d156330e018184991dd02e75009}{
\index{mainwindow.cpp@{mainwindow.cpp}!g@{g}}
\index{g@{g}!mainwindow.cpp@{mainwindow.cpp}}
\subsubsection[g]{\setlength{\rightskip}{0pt plus 5cm}int \hyperlink{mainwindow_8cpp_2e153d156330e018184991dd02e75009}{g}\mbox{[}NUMBEROFROBOTS\mbox{]}}}
\label{mainwindow_8cpp_2e153d156330e018184991dd02e75009}


\hypertarget{mainwindow_8cpp_b6a9976891cb20971afedcb268e596a0}{
\index{mainwindow.cpp@{mainwindow.cpp}!goButton@{goButton}}
\index{goButton@{goButton}!mainwindow.cpp@{mainwindow.cpp}}
\subsubsection[goButton]{\setlength{\rightskip}{0pt plus 5cm}QPush\-Button$\ast$ \hyperlink{mainwindow_8cpp_b6a9976891cb20971afedcb268e596a0}{go\-Button}}}
\label{mainwindow_8cpp_b6a9976891cb20971afedcb268e596a0}


\hypertarget{mainwindow_8cpp_5b3043aacdbae62352f5993a33c4a4b7}{
\index{mainwindow.cpp@{mainwindow.cpp}!imageLabel@{imageLabel}}
\index{imageLabel@{imageLabel}!mainwindow.cpp@{mainwindow.cpp}}
\subsubsection[imageLabel]{\setlength{\rightskip}{0pt plus 5cm}QLabel$\ast$ \hyperlink{mainwindow_8cpp_5b3043aacdbae62352f5993a33c4a4b7}{image\-Label}}}
\label{mainwindow_8cpp_5b3043aacdbae62352f5993a33c4a4b7}


\hypertarget{mainwindow_8cpp_3fbb8173a9510be37c564f103a543aaa}{
\index{mainwindow.cpp@{mainwindow.cpp}!info1@{info1}}
\index{info1@{info1}!mainwindow.cpp@{mainwindow.cpp}}
\subsubsection[info1]{\setlength{\rightskip}{0pt plus 5cm}QText\-Edit$\ast$ \hyperlink{mainwindow_8cpp_3fbb8173a9510be37c564f103a543aaa}{info1}}}
\label{mainwindow_8cpp_3fbb8173a9510be37c564f103a543aaa}


\hypertarget{mainwindow_8cpp_850f93f80a2433d5fadf30d5e5e866f4}{
\index{mainwindow.cpp@{mainwindow.cpp}!info2@{info2}}
\index{info2@{info2}!mainwindow.cpp@{mainwindow.cpp}}
\subsubsection[info2]{\setlength{\rightskip}{0pt plus 5cm}QText\-Edit$\ast$ \hyperlink{mainwindow_8cpp_850f93f80a2433d5fadf30d5e5e866f4}{info2}}}
\label{mainwindow_8cpp_850f93f80a2433d5fadf30d5e5e866f4}


\hypertarget{mainwindow_8cpp_0b2e8c7f76df48129f994ecc46d5c66c}{
\index{mainwindow.cpp@{mainwindow.cpp}!message@{message}}
\index{message@{message}!mainwindow.cpp@{mainwindow.cpp}}
\subsubsection[message]{\setlength{\rightskip}{0pt plus 5cm}char$\ast$ \hyperlink{mainwindow_8cpp_0b2e8c7f76df48129f994ecc46d5c66c}{message}}}
\label{mainwindow_8cpp_0b2e8c7f76df48129f994ecc46d5c66c}


\hypertarget{mainwindow_8cpp_84351fcfbbed5a81cce783cb68e2eaf6}{
\index{mainwindow.cpp@{mainwindow.cpp}!openPortButton@{openPortButton}}
\index{openPortButton@{openPortButton}!mainwindow.cpp@{mainwindow.cpp}}
\subsubsection[openPortButton]{\setlength{\rightskip}{0pt plus 5cm}QPush\-Button$\ast$ \hyperlink{mainwindow_8cpp_84351fcfbbed5a81cce783cb68e2eaf6}{open\-Port\-Button}}}
\label{mainwindow_8cpp_84351fcfbbed5a81cce783cb68e2eaf6}


\hypertarget{mainwindow_8cpp_00eeca5cdf915ad3d1fd8dbf28ab156d}{
\index{mainwindow.cpp@{mainwindow.cpp}!portacom@{portacom}}
\index{portacom@{portacom}!mainwindow.cpp@{mainwindow.cpp}}
\subsubsection[portacom]{\setlength{\rightskip}{0pt plus 5cm}char$\ast$ \hyperlink{mainwindow_8cpp_00eeca5cdf915ad3d1fd8dbf28ab156d}{portacom}}}
\label{mainwindow_8cpp_00eeca5cdf915ad3d1fd8dbf28ab156d}


\hypertarget{mainwindow_8cpp_6b10b2ae96c856016dc4e92021ea50af}{
\index{mainwindow.cpp@{mainwindow.cpp}!portHandle@{portHandle}}
\index{portHandle@{portHandle}!mainwindow.cpp@{mainwindow.cpp}}
\subsubsection[portHandle]{\setlength{\rightskip}{0pt plus 5cm}HANDLE \hyperlink{mainwindow_8cpp_6b10b2ae96c856016dc4e92021ea50af}{port\-Handle}}}
\label{mainwindow_8cpp_6b10b2ae96c856016dc4e92021ea50af}


\hypertarget{mainwindow_8cpp_348a8673070b1d1e2f91feb1149cb76b}{
\index{mainwindow.cpp@{mainwindow.cpp}!pos_x@{pos\_\-x}}
\index{pos_x@{pos\_\-x}!mainwindow.cpp@{mainwindow.cpp}}
\subsubsection[pos\_\-x]{\setlength{\rightskip}{0pt plus 5cm}int \hyperlink{mainwindow_8cpp_348a8673070b1d1e2f91feb1149cb76b}{pos\_\-x}\mbox{[}NUMBEROFROBOTS\mbox{]}}}
\label{mainwindow_8cpp_348a8673070b1d1e2f91feb1149cb76b}


\hypertarget{mainwindow_8cpp_f803113f8db598bbb5cc7cb55c53fd72}{
\index{mainwindow.cpp@{mainwindow.cpp}!pos_y@{pos\_\-y}}
\index{pos_y@{pos\_\-y}!mainwindow.cpp@{mainwindow.cpp}}
\subsubsection[pos\_\-y]{\setlength{\rightskip}{0pt plus 5cm}int \hyperlink{mainwindow_8cpp_f803113f8db598bbb5cc7cb55c53fd72}{pos\_\-y}\mbox{[}NUMBEROFROBOTS\mbox{]}}}
\label{mainwindow_8cpp_f803113f8db598bbb5cc7cb55c53fd72}


\hypertarget{mainwindow_8cpp_a194f1a2c336647d7165309ff921f576}{
\index{mainwindow.cpp@{mainwindow.cpp}!r@{r}}
\index{r@{r}!mainwindow.cpp@{mainwindow.cpp}}
\subsubsection[r]{\setlength{\rightskip}{0pt plus 5cm}int \hyperlink{mainwindow_8cpp_a194f1a2c336647d7165309ff921f576}{r}\mbox{[}NUMBEROFROBOTS\mbox{]}}}
\label{mainwindow_8cpp_a194f1a2c336647d7165309ff921f576}


\hypertarget{mainwindow_8cpp_65bf9d32222c82b489f99fcb4f892edd}{
\index{mainwindow.cpp@{mainwindow.cpp}!recvbuffer@{recvbuffer}}
\index{recvbuffer@{recvbuffer}!mainwindow.cpp@{mainwindow.cpp}}
\subsubsection[recvbuffer]{\setlength{\rightskip}{0pt plus 5cm}char \hyperlink{mainwindow_8cpp_65bf9d32222c82b489f99fcb4f892edd}{recvbuffer}\mbox{[}RECVBUFFERSIZE\mbox{]}}}
\label{mainwindow_8cpp_65bf9d32222c82b489f99fcb4f892edd}


\hypertarget{mainwindow_8cpp_ebf8093f61c93230d92191726d5bc846}{
\index{mainwindow.cpp@{mainwindow.cpp}!robot@{robot}}
\index{robot@{robot}!mainwindow.cpp@{mainwindow.cpp}}
\subsubsection[robot]{\setlength{\rightskip}{0pt plus 5cm}int \hyperlink{mainwindow_8cpp_ebf8093f61c93230d92191726d5bc846}{robot}}}
\label{mainwindow_8cpp_ebf8093f61c93230d92191726d5bc846}


\hypertarget{mainwindow_8cpp_e31aae5c36d2da2116798bb42bf14d73}{
\index{mainwindow.cpp@{mainwindow.cpp}!robotchanged@{robotchanged}}
\index{robotchanged@{robotchanged}!mainwindow.cpp@{mainwindow.cpp}}
\subsubsection[robotchanged]{\setlength{\rightskip}{0pt plus 5cm}bool \hyperlink{mainwindow_8cpp_e31aae5c36d2da2116798bb42bf14d73}{robotchanged}}}
\label{mainwindow_8cpp_e31aae5c36d2da2116798bb42bf14d73}


\hypertarget{mainwindow_8cpp_852e2307ac8c879169cd0f76ed135c1b}{
\index{mainwindow.cpp@{mainwindow.cpp}!setPosButton@{setPosButton}}
\index{setPosButton@{setPosButton}!mainwindow.cpp@{mainwindow.cpp}}
\subsubsection[setPosButton]{\setlength{\rightskip}{0pt plus 5cm}QPush\-Button$\ast$ \hyperlink{mainwindow_8cpp_852e2307ac8c879169cd0f76ed135c1b}{set\-Pos\-Button}}}
\label{mainwindow_8cpp_852e2307ac8c879169cd0f76ed135c1b}


\hypertarget{mainwindow_8cpp_a68dc4dea54cd443bda2a53f9b0621cd}{
\index{mainwindow.cpp@{mainwindow.cpp}!startButton@{startButton}}
\index{startButton@{startButton}!mainwindow.cpp@{mainwindow.cpp}}
\subsubsection[startButton]{\setlength{\rightskip}{0pt plus 5cm}QPush\-Button$\ast$ \hyperlink{mainwindow_8cpp_a68dc4dea54cd443bda2a53f9b0621cd}{start\-Button}}}
\label{mainwindow_8cpp_a68dc4dea54cd443bda2a53f9b0621cd}


\hypertarget{mainwindow_8cpp_12a122b59e7d02ed91703cb0568faa3c}{
\index{mainwindow.cpp@{mainwindow.cpp}!stopButton@{stopButton}}
\index{stopButton@{stopButton}!mainwindow.cpp@{mainwindow.cpp}}
\subsubsection[stopButton]{\setlength{\rightskip}{0pt plus 5cm}QPush\-Button$\ast$ \hyperlink{mainwindow_8cpp_12a122b59e7d02ed91703cb0568faa3c}{stop\-Button}}}
\label{mainwindow_8cpp_12a122b59e7d02ed91703cb0568faa3c}


\hypertarget{mainwindow_8cpp_288b834f8b483958be715bc77558bf46}{
\index{mainwindow.cpp@{mainwindow.cpp}!str@{str}}
\index{str@{str}!mainwindow.cpp@{mainwindow.cpp}}
\subsubsection[str]{\setlength{\rightskip}{0pt plus 5cm}QString \hyperlink{mainwindow_8cpp_288b834f8b483958be715bc77558bf46}{str}}}
\label{mainwindow_8cpp_288b834f8b483958be715bc77558bf46}



\hypertarget{mainwindow_8h}{
\section{Riferimenti per il file mainwindow.h}
\label{mainwindow_8h}\index{mainwindow.h@{mainwindow.h}}
}
{\tt \#include $<$QMain\-Window$>$}\par
{\tt \#include $<$iostream$>$}\par
{\tt \#include $<$fstream$>$}\par
{\tt \#include $<$windows.h$>$}\par
{\tt \#include $<$stdio.h$>$}\par
{\tt \#include \char`\"{}Serial\-Port.h\char`\"{}}\par
{\tt \#include \char`\"{}QPush\-Button.h\char`\"{}}\par
\subsection*{Strutture dati}
\begin{CompactItemize}
\item 
class \hyperlink{classMainWindow}{Main\-Window}
\end{CompactItemize}
\subsection*{Funzioni}
\begin{CompactItemize}
\item 
void \hyperlink{mainwindow_8h_69a50a89c86348f01cc23a429685e9dd}{define\-Robot\-Colors} ()
\item 
void \hyperlink{mainwindow_8h_046465f393d0876a74f11c66c1c95fa4}{listeningthread} ()
\end{CompactItemize}


\subsection{Documentazione delle funzioni}
\hypertarget{mainwindow_8h_69a50a89c86348f01cc23a429685e9dd}{
\index{mainwindow.h@{mainwindow.h}!defineRobotColors@{defineRobotColors}}
\index{defineRobotColors@{defineRobotColors}!mainwindow.h@{mainwindow.h}}
\subsubsection[defineRobotColors]{\setlength{\rightskip}{0pt plus 5cm}void define\-Robot\-Colors ()}}
\label{mainwindow_8h_69a50a89c86348f01cc23a429685e9dd}


Customize the robot indicator's color. \hypertarget{mainwindow_8h_046465f393d0876a74f11c66c1c95fa4}{
\index{mainwindow.h@{mainwindow.h}!listeningthread@{listeningthread}}
\index{listeningthread@{listeningthread}!mainwindow.h@{mainwindow.h}}
\subsubsection[listeningthread]{\setlength{\rightskip}{0pt plus 5cm}void listeningthread ()}}
\label{mainwindow_8h_046465f393d0876a74f11c66c1c95fa4}



\hypertarget{SerialPort_8cpp}{
\section{Riferimenti per il file Serial\-Port.cpp}
\label{SerialPort_8cpp}\index{SerialPort.cpp@{SerialPort.cpp}}
}
{\tt \#include \char`\"{}Serial\-Port.h\char`\"{}}\par
{\tt \#include \char`\"{}mainwindow.h\char`\"{}}\par
\subsection*{Definizioni}
\begin{CompactItemize}
\item 
\#define \hyperlink{SerialPort_8cpp_515119526a2e89ab018116ac95b628c0}{RECVBUFFERSIZE}~512
\end{CompactItemize}
\subsection*{Funzioni}
\begin{CompactItemize}
\item 
int \hyperlink{SerialPort_8cpp_18feec80309c589013c02a64d88c7c26}{close\-Port} (HANDLE $\ast$porth)
\item 
int \hyperlink{SerialPort_8cpp_ea3d3d746e0bfe0e0f61bcd991329be7}{open\-Port} (HANDLE $\ast$porth, char $\ast$\hyperlink{mainwindow_8cpp_00eeca5cdf915ad3d1fd8dbf28ab156d}{portacom})
\item 
void \hyperlink{SerialPort_8cpp_f13ac658417232fff936412dd02c7258}{receivedata} (HANDLE $\ast$porth, char $\ast$\hyperlink{mainwindow_8cpp_65bf9d32222c82b489f99fcb4f892edd}{recvbuffer})
\item 
bool \hyperlink{SerialPort_8cpp_8f9bf1e54fe3f7489aa15f282455f068}{senddata} (HANDLE $\ast$porth, const char $\ast$d, double size)
\end{CompactItemize}
\subsection*{Variabili}
\begin{CompactItemize}
\item 
COMMTIMEOUTS \hyperlink{SerialPort_8cpp_c6a3a6a1b69f5ce0b07b1959743d00ed}{cmt}
\item 
DCB \hyperlink{SerialPort_8cpp_9954a8d34e6014302ae1d99cf5714cc9}{dcb}
\item 
\hyperlink{classMainWindow}{Main\-Window} $\ast$ \hyperlink{SerialPort_8cpp_b6705fab832298411ae13ace059dd583}{main\-Window}
\item 
DWORD \hyperlink{SerialPort_8cpp_8c7659aa4473fb28708802fb35b9eef7}{read}
\item 
DWORD \hyperlink{SerialPort_8cpp_499ee21e69b1456284a39e1fe7996d65}{write}
\end{CompactItemize}


\subsection{Documentazione delle definizioni}
\hypertarget{SerialPort_8cpp_515119526a2e89ab018116ac95b628c0}{
\index{SerialPort.cpp@{Serial\-Port.cpp}!RECVBUFFERSIZE@{RECVBUFFERSIZE}}
\index{RECVBUFFERSIZE@{RECVBUFFERSIZE}!SerialPort.cpp@{Serial\-Port.cpp}}
\subsubsection[RECVBUFFERSIZE]{\setlength{\rightskip}{0pt plus 5cm}\#define RECVBUFFERSIZE~512}}
\label{SerialPort_8cpp_515119526a2e89ab018116ac95b628c0}




\subsection{Documentazione delle funzioni}
\hypertarget{SerialPort_8cpp_18feec80309c589013c02a64d88c7c26}{
\index{SerialPort.cpp@{Serial\-Port.cpp}!closePort@{closePort}}
\index{closePort@{closePort}!SerialPort.cpp@{Serial\-Port.cpp}}
\subsubsection[closePort]{\setlength{\rightskip}{0pt plus 5cm}int close\-Port (HANDLE $\ast$ {\em porth})}}
\label{SerialPort_8cpp_18feec80309c589013c02a64d88c7c26}


Close the COM port \begin{Desc}
\item[Parametri:]
\begin{description}
\item[{\em porth}]The handle \end{description}
\end{Desc}
\hypertarget{SerialPort_8cpp_ea3d3d746e0bfe0e0f61bcd991329be7}{
\index{SerialPort.cpp@{Serial\-Port.cpp}!openPort@{openPort}}
\index{openPort@{openPort}!SerialPort.cpp@{Serial\-Port.cpp}}
\subsubsection[openPort]{\setlength{\rightskip}{0pt plus 5cm}int open\-Port (HANDLE $\ast$ {\em porth}, char $\ast$ {\em portacom})}}
\label{SerialPort_8cpp_ea3d3d746e0bfe0e0f61bcd991329be7}


Create the handle and open the COM port \begin{Desc}
\item[Parametri:]
\begin{description}
\item[{\em porth}]The handle \item[{\em portacom}]The COM port you have to open \end{description}
\end{Desc}
\hypertarget{SerialPort_8cpp_f13ac658417232fff936412dd02c7258}{
\index{SerialPort.cpp@{Serial\-Port.cpp}!receivedata@{receivedata}}
\index{receivedata@{receivedata}!SerialPort.cpp@{Serial\-Port.cpp}}
\subsubsection[receivedata]{\setlength{\rightskip}{0pt plus 5cm}void receivedata (HANDLE $\ast$ {\em porth}, char $\ast$ {\em recvbuffer})}}
\label{SerialPort_8cpp_f13ac658417232fff936412dd02c7258}


Receive (reading the COM port) \hypertarget{SerialPort_8cpp_8f9bf1e54fe3f7489aa15f282455f068}{
\index{SerialPort.cpp@{Serial\-Port.cpp}!senddata@{senddata}}
\index{senddata@{senddata}!SerialPort.cpp@{Serial\-Port.cpp}}
\subsubsection[senddata]{\setlength{\rightskip}{0pt plus 5cm}bool senddata (HANDLE $\ast$ {\em porth}, const char $\ast$ {\em d}, double {\em size})}}
\label{SerialPort_8cpp_8f9bf1e54fe3f7489aa15f282455f068}


Send (writing on COM port) \begin{Desc}
\item[Parametri:]
\begin{description}
\item[{\em porth}]The handle \item[{\em d}]the string that you have to send \item[{\em size}]the size of the string \end{description}
\end{Desc}


\subsection{Documentazione delle variabili}
\hypertarget{SerialPort_8cpp_c6a3a6a1b69f5ce0b07b1959743d00ed}{
\index{SerialPort.cpp@{Serial\-Port.cpp}!cmt@{cmt}}
\index{cmt@{cmt}!SerialPort.cpp@{Serial\-Port.cpp}}
\subsubsection[cmt]{\setlength{\rightskip}{0pt plus 5cm}COMMTIMEOUTS \hyperlink{SerialPort_8cpp_c6a3a6a1b69f5ce0b07b1959743d00ed}{cmt}}}
\label{SerialPort_8cpp_c6a3a6a1b69f5ce0b07b1959743d00ed}


\hypertarget{SerialPort_8cpp_9954a8d34e6014302ae1d99cf5714cc9}{
\index{SerialPort.cpp@{Serial\-Port.cpp}!dcb@{dcb}}
\index{dcb@{dcb}!SerialPort.cpp@{Serial\-Port.cpp}}
\subsubsection[dcb]{\setlength{\rightskip}{0pt plus 5cm}DCB \hyperlink{SerialPort_8cpp_9954a8d34e6014302ae1d99cf5714cc9}{dcb}}}
\label{SerialPort_8cpp_9954a8d34e6014302ae1d99cf5714cc9}


\hypertarget{SerialPort_8cpp_b6705fab832298411ae13ace059dd583}{
\index{SerialPort.cpp@{Serial\-Port.cpp}!mainWindow@{mainWindow}}
\index{mainWindow@{mainWindow}!SerialPort.cpp@{Serial\-Port.cpp}}
\subsubsection[mainWindow]{\setlength{\rightskip}{0pt plus 5cm}\hyperlink{classMainWindow}{Main\-Window}$\ast$ \hyperlink{SerialPort_8cpp_b6705fab832298411ae13ace059dd583}{main\-Window}}}
\label{SerialPort_8cpp_b6705fab832298411ae13ace059dd583}


Title: Seril\-Port.cpp 

Description: Implementation of the \hyperlink{classSerialPort}{Serial\-Port} class. \begin{Desc}
\item[Autore:]Andrea Maglie (Matr.456188) \end{Desc}
\begin{Desc}
\item[Versione:]1.0 \end{Desc}
\hypertarget{SerialPort_8cpp_8c7659aa4473fb28708802fb35b9eef7}{
\index{SerialPort.cpp@{Serial\-Port.cpp}!read@{read}}
\index{read@{read}!SerialPort.cpp@{Serial\-Port.cpp}}
\subsubsection[read]{\setlength{\rightskip}{0pt plus 5cm}DWORD \hyperlink{SerialPort_8cpp_8c7659aa4473fb28708802fb35b9eef7}{read}}}
\label{SerialPort_8cpp_8c7659aa4473fb28708802fb35b9eef7}


\hypertarget{SerialPort_8cpp_499ee21e69b1456284a39e1fe7996d65}{
\index{SerialPort.cpp@{Serial\-Port.cpp}!write@{write}}
\index{write@{write}!SerialPort.cpp@{Serial\-Port.cpp}}
\subsubsection[write]{\setlength{\rightskip}{0pt plus 5cm}DWORD \hyperlink{SerialPort_8cpp_499ee21e69b1456284a39e1fe7996d65}{write}}}
\label{SerialPort_8cpp_499ee21e69b1456284a39e1fe7996d65}



\hypertarget{SerialPort_8h}{
\section{Riferimenti per il file Serial\-Port.h}
\label{SerialPort_8h}\index{SerialPort.h@{SerialPort.h}}
}
{\tt \#include $<$iostream$>$}\par
{\tt \#include $<$fstream$>$}\par
{\tt \#include $<$windows.h$>$}\par
{\tt \#include $<$stdio.h$>$}\par
{\tt \#include $<$string.h$>$}\par
{\tt \#include $<$QThread$>$}\par
{\tt \#include \char`\"{}mainwindow.h\char`\"{}}\par
\subsection*{Strutture dati}
\begin{CompactItemize}
\item 
class \hyperlink{classSerialPort}{Serial\-Port}
\end{CompactItemize}
\subsection*{Funzioni}
\begin{CompactItemize}
\item 
int \hyperlink{SerialPort_8h_18feec80309c589013c02a64d88c7c26}{close\-Port} (HANDLE $\ast$porth)
\item 
DWORD WINAPI \hyperlink{SerialPort_8h_3ec429244ead8f4eb9f75e05fbfee35e}{funzione} (LPDWORD $\ast$lpdw\-Param)
\item 
int \hyperlink{SerialPort_8h_ea3d3d746e0bfe0e0f61bcd991329be7}{open\-Port} (HANDLE $\ast$porth, char $\ast$\hyperlink{mainwindow_8cpp_00eeca5cdf915ad3d1fd8dbf28ab156d}{portacom})
\item 
void \hyperlink{SerialPort_8h_f13ac658417232fff936412dd02c7258}{receivedata} (HANDLE $\ast$porth, char $\ast$\hyperlink{mainwindow_8cpp_65bf9d32222c82b489f99fcb4f892edd}{recvbuffer})
\item 
bool \hyperlink{SerialPort_8h_8f9bf1e54fe3f7489aa15f282455f068}{senddata} (HANDLE $\ast$porth, const char $\ast$d, double size)
\end{CompactItemize}


\subsection{Documentazione delle funzioni}
\hypertarget{SerialPort_8h_18feec80309c589013c02a64d88c7c26}{
\index{SerialPort.h@{Serial\-Port.h}!closePort@{closePort}}
\index{closePort@{closePort}!SerialPort.h@{Serial\-Port.h}}
\subsubsection[closePort]{\setlength{\rightskip}{0pt plus 5cm}int close\-Port (HANDLE $\ast$ {\em porth})}}
\label{SerialPort_8h_18feec80309c589013c02a64d88c7c26}


Close the COM port \begin{Desc}
\item[Parametri:]
\begin{description}
\item[{\em porth}]The handle \end{description}
\end{Desc}
\hypertarget{SerialPort_8h_3ec429244ead8f4eb9f75e05fbfee35e}{
\index{SerialPort.h@{Serial\-Port.h}!funzione@{funzione}}
\index{funzione@{funzione}!SerialPort.h@{Serial\-Port.h}}
\subsubsection[funzione]{\setlength{\rightskip}{0pt plus 5cm}DWORD WINAPI funzione (LPDWORD $\ast$ {\em lpdw\-Param})}}
\label{SerialPort_8h_3ec429244ead8f4eb9f75e05fbfee35e}


\hypertarget{SerialPort_8h_ea3d3d746e0bfe0e0f61bcd991329be7}{
\index{SerialPort.h@{Serial\-Port.h}!openPort@{openPort}}
\index{openPort@{openPort}!SerialPort.h@{Serial\-Port.h}}
\subsubsection[openPort]{\setlength{\rightskip}{0pt plus 5cm}int open\-Port (HANDLE $\ast$ {\em porth}, char $\ast$ {\em portacom})}}
\label{SerialPort_8h_ea3d3d746e0bfe0e0f61bcd991329be7}


Create the handle and open the COM port \begin{Desc}
\item[Parametri:]
\begin{description}
\item[{\em porth}]The handle \item[{\em portacom}]The COM port you have to open \end{description}
\end{Desc}
\hypertarget{SerialPort_8h_f13ac658417232fff936412dd02c7258}{
\index{SerialPort.h@{Serial\-Port.h}!receivedata@{receivedata}}
\index{receivedata@{receivedata}!SerialPort.h@{Serial\-Port.h}}
\subsubsection[receivedata]{\setlength{\rightskip}{0pt plus 5cm}void receivedata (HANDLE $\ast$ {\em porth}, char $\ast$ {\em recvbuffer})}}
\label{SerialPort_8h_f13ac658417232fff936412dd02c7258}


Receive (reading the COM port) \hypertarget{SerialPort_8h_8f9bf1e54fe3f7489aa15f282455f068}{
\index{SerialPort.h@{Serial\-Port.h}!senddata@{senddata}}
\index{senddata@{senddata}!SerialPort.h@{Serial\-Port.h}}
\subsubsection[senddata]{\setlength{\rightskip}{0pt plus 5cm}bool senddata (HANDLE $\ast$ {\em porth}, const char $\ast$ {\em d}, double {\em size})}}
\label{SerialPort_8h_8f9bf1e54fe3f7489aa15f282455f068}


Send (writing on COM port) \begin{Desc}
\item[Parametri:]
\begin{description}
\item[{\em porth}]The handle \item[{\em d}]the string that you have to send \item[{\em size}]the size of the string \end{description}
\end{Desc}

\printindex
\end{document}
